\batchmode
\documentclass[a4paper,12pt]{book}
\RequirePackage{ifthen}


\usepackage{graphicx}
\usepackage{tabularx}
\usepackage{a4wide}
\usepackage{amsfonts}
\usepackage{amsmath}
\usepackage[usenames,dvipsnames]{color}
\usepackage{html}
\usepackage{verbatim}
\usepackage{pytex}
\usepackage{xspace}

%
\providecommand{\n}{\\n}%
\providecommand{\CasADi}{\texttt{CasADi}\xspace}%
\providecommand{\trace}[1]{\text{tr}(#1)}%
\providecommand{\T}{\text{T}}%
\providecommand{\lb}{\text{lb}}%
\providecommand{\ub}{\text{ub}}%
\providecommand{\python}[1]{\lstinline[language=Python]{#1}}%
\providecommand{\cxx}[1]{\lstinline[language=C++]{#1}} 

%
\providecommand{\currentversion}{currentversionplaceholder} 

\newcounter{pytexcount} 
\newcounter{pytexsubcount} 
\newcounter{pytexlinecountstart} 
\newcounter{pytexlinecountend} 

\setcounter{pytexcount}{0}
\setcounter{pytexlinecountstart}{0}
\setcounter{pytexlinecountend}{0}



%
\newenvironment{pytexTemplate}[1]{
\begin{rawhtml}

<div style="display:none">\end{rawhtml}

}{
\begin{rawhtml}

</div>\end{rawhtml}

} 

%
\providecommand{\pytexStart}[1]{
  \addtocounter{pytexcount}{1}%   'pytexcount'++
  \setcounter{pytexsubcount}{0}%  reset 'pytexsubcount'
} 


\renewenvironment{pytex}{\addtocounter{pytexsubcount}{1}%                                             'pytexsubcount'++
  \begin{rawhtml}

<div style="color: black; background-color: \#b9c8db;  border-style: dotted; border-width: 1px; padding:2px;padding-left:1em" >
<pre>\end{rawhtml}

}%
{\begin{rawhtml}

</pre>
</div>
<div style="color: black; background-color: \#fffff;  border-style: solid; border-width: 1px; padding:2px;padding-left:1em;margin-left:1em;" >\end{rawhtml}
%
\verbatiminputeval{pytex_\alph{pytexcount}_\arabic{pytexsubcount}.log}%
\begin{rawhtml}

</div>\end{rawhtml}

}
\renewenvironment{pytexoutput}{\addtocounter{pytexsubcount}{1}%                                             'pytexsubcount'++
  \begin{rawhtml}

<div style="display:none">
<pre>\end{rawhtml}

}%
{\begin{rawhtml}

</pre>
</div>
<div style="color: black; background-color: \#fffff;  border-style: solid; border-width: 1px; padding:2px;padding-left:1em;margin-left:1em;" >\end{rawhtml}
%
\verbatiminputeval{pytex_\alph{pytexcount}_\arabic{pytexsubcount}.log}%
\begin{rawhtml}

</div>\end{rawhtml}

}%
\providecommand{\codebegin}{
\begin{rawhtml}

<div style="color: black; background-color: \#b9c8db;  border-style: dotted; border-width: 1px;padding:2px;padding-left:1em" >\end{rawhtml}

}%
\providecommand{\codeend}{
\begin{rawhtml}

</div>\end{rawhtml}

}%
\providecommand{\codebegin}{


}%
\providecommand{\codeend}{


} 


\author{Joel Andersson \and Joris Gillis \and Moritz Diehl}
\title{User Documentation for \texttt{CasADi}\xspace vcurrentversionplaceholder}



\pagecolor[gray]{.7}

\usepackage[]{inputenc}



\makeatletter

\makeatletter
\count@=\the\catcode`\_ \catcode`\_=8 
\newenvironment{tex2html_wrap}{}{}%
\catcode`\<=12\catcode`\_=\count@
\newcommand{\providedcommand}[1]{\expandafter\providecommand\csname #1\endcsname}%
\newcommand{\renewedcommand}[1]{\expandafter\providecommand\csname #1\endcsname{}%
  \expandafter\renewcommand\csname #1\endcsname}%
\newcommand{\newedenvironment}[1]{\newenvironment{#1}{}{}\renewenvironment{#1}}%
\let\newedcommand\renewedcommand
\let\renewedenvironment\newedenvironment
\makeatother
\let\mathon=$
\let\mathoff=$
\ifx\AtBeginDocument\undefined \newcommand{\AtBeginDocument}[1]{}\fi
\newbox\sizebox
\setlength{\hoffset}{0pt}\setlength{\voffset}{0pt}
\addtolength{\textheight}{\footskip}\setlength{\footskip}{0pt}
\addtolength{\textheight}{\topmargin}\setlength{\topmargin}{0pt}
\addtolength{\textheight}{\headheight}\setlength{\headheight}{0pt}
\addtolength{\textheight}{\headsep}\setlength{\headsep}{0pt}
\setlength{\textwidth}{349pt}
\newwrite\lthtmlwrite
\makeatletter
\let\realnormalsize=\normalsize
\global\topskip=2sp
\def\preveqno{}\let\real@float=\@float \let\realend@float=\end@float
\def\@float{\let\@savefreelist\@freelist\real@float}
\def\liih@math{\ifmmode$\else\bad@math\fi}
\def\end@float{\realend@float\global\let\@freelist\@savefreelist}
\let\real@dbflt=\@dbflt \let\end@dblfloat=\end@float
\let\@largefloatcheck=\relax
\let\if@boxedmulticols=\iftrue
\def\@dbflt{\let\@savefreelist\@freelist\real@dbflt}
\def\adjustnormalsize{\def\normalsize{\mathsurround=0pt \realnormalsize
 \parindent=0pt\abovedisplayskip=0pt\belowdisplayskip=0pt}%
 \def\phantompar{\csname par\endcsname}\normalsize}%
\def\lthtmltypeout#1{{\let\protect\string \immediate\write\lthtmlwrite{#1}}}%
\newcommand\lthtmlhboxmathA{\adjustnormalsize\setbox\sizebox=\hbox\bgroup\kern.05em }%
\newcommand\lthtmlhboxmathB{\adjustnormalsize\setbox\sizebox=\hbox to\hsize\bgroup\hfill }%
\newcommand\lthtmlvboxmathA{\adjustnormalsize\setbox\sizebox=\vbox\bgroup %
 \let\ifinner=\iffalse \let\)\liih@math }%
\newcommand\lthtmlboxmathZ{\@next\next\@currlist{}{\def\next{\voidb@x}}%
 \expandafter\box\next\egroup}%
\newcommand\lthtmlmathtype[1]{\gdef\lthtmlmathenv{#1}}%
\newcommand\lthtmllogmath{\dimen0\ht\sizebox \advance\dimen0\dp\sizebox
  \ifdim\dimen0>.95\vsize
   \lthtmltypeout{%
*** image for \lthtmlmathenv\space is too tall at \the\dimen0, reducing to .95 vsize ***}%
   \ht\sizebox.95\vsize \dp\sizebox\z@ \fi
  \lthtmltypeout{l2hSize %
:\lthtmlmathenv:\the\ht\sizebox::\the\dp\sizebox::\the\wd\sizebox.\preveqno}}%
\newcommand\lthtmlfigureA[1]{\let\@savefreelist\@freelist
       \lthtmlmathtype{#1}\lthtmlvboxmathA}%
\newcommand\lthtmlpictureA{\bgroup\catcode`\_=8 \lthtmlpictureB}%
\newcommand\lthtmlpictureB[1]{\lthtmlmathtype{#1}\egroup
       \let\@savefreelist\@freelist \lthtmlhboxmathB}%
\newcommand\lthtmlpictureZ[1]{\hfill\lthtmlfigureZ}%
\newcommand\lthtmlfigureZ{\lthtmlboxmathZ\lthtmllogmath\copy\sizebox
       \global\let\@freelist\@savefreelist}%
\newcommand\lthtmldisplayA{\bgroup\catcode`\_=8 \lthtmldisplayAi}%
\newcommand\lthtmldisplayAi[1]{\lthtmlmathtype{#1}\egroup\lthtmlvboxmathA}%
\newcommand\lthtmldisplayB[1]{\edef\preveqno{(\theequation)}%
  \lthtmldisplayA{#1}\let\@eqnnum\relax}%
\newcommand\lthtmldisplayZ{\lthtmlboxmathZ\lthtmllogmath\lthtmlsetmath}%
\newcommand\lthtmlinlinemathA{\bgroup\catcode`\_=8 \lthtmlinlinemathB}
\newcommand\lthtmlinlinemathB[1]{\lthtmlmathtype{#1}\egroup\lthtmlhboxmathA
  \vrule height1.5ex width0pt }%
\newcommand\lthtmlinlineA{\bgroup\catcode`\_=8 \lthtmlinlineB}%
\newcommand\lthtmlinlineB[1]{\lthtmlmathtype{#1}\egroup\lthtmlhboxmathA}%
\newcommand\lthtmlinlineZ{\egroup\expandafter\ifdim\dp\sizebox>0pt %
  \expandafter\centerinlinemath\fi\lthtmllogmath\lthtmlsetinline}
\newcommand\lthtmlinlinemathZ{\egroup\expandafter\ifdim\dp\sizebox>0pt %
  \expandafter\centerinlinemath\fi\lthtmllogmath\lthtmlsetmath}
\newcommand\lthtmlindisplaymathZ{\egroup %
  \centerinlinemath\lthtmllogmath\lthtmlsetmath}
\def\lthtmlsetinline{\hbox{\vrule width.1em \vtop{\vbox{%
  \kern.1em\copy\sizebox}\ifdim\dp\sizebox>0pt\kern.1em\else\kern.3pt\fi
  \ifdim\hsize>\wd\sizebox \hrule depth1pt\fi}}}
\def\lthtmlsetmath{\hbox{\vrule width.1em\kern-.05em\vtop{\vbox{%
  \kern.1em\kern0.8 pt\hbox{\hglue.17em\copy\sizebox\hglue0.8 pt}}\kern.3pt%
  \ifdim\dp\sizebox>0pt\kern.1em\fi \kern0.8 pt%
  \ifdim\hsize>\wd\sizebox \hrule depth1pt\fi}}}
\def\centerinlinemath{%
  \dimen1=\ifdim\ht\sizebox<\dp\sizebox \dp\sizebox\else\ht\sizebox\fi
  \advance\dimen1by.5pt \vrule width0pt height\dimen1 depth\dimen1 
 \dp\sizebox=\dimen1\ht\sizebox=\dimen1\relax}

\def\lthtmlcheckvsize{\ifdim\ht\sizebox<\vsize 
  \ifdim\wd\sizebox<\hsize\expandafter\hfill\fi \expandafter\vfill
  \else\expandafter\vss\fi}%
\providecommand{\selectlanguage}[1]{}%
\makeatletter \tracingstats = 1 
\providecommand{\Beta}{\textrm{B}}
\providecommand{\Mu}{\textrm{M}}
\providecommand{\Kappa}{\textrm{K}}
\providecommand{\Rho}{\textrm{R}}
\providecommand{\Epsilon}{\textrm{E}}
\providecommand{\Chi}{\textrm{X}}
\providecommand{\Iota}{\textrm{J}}
\providecommand{\omicron}{\textrm{o}}
\providecommand{\Zeta}{\textrm{Z}}
\providecommand{\Eta}{\textrm{H}}
\providecommand{\Omicron}{\textrm{O}}
\providecommand{\Nu}{\textrm{N}}
\providecommand{\Tau}{\textrm{T}}
\providecommand{\Alpha}{\textrm{A}}


\begin{document}
\pagestyle{empty}\thispagestyle{empty}\lthtmltypeout{}%
\lthtmltypeout{latex2htmlLength hsize=\the\hsize}\lthtmltypeout{}%
\lthtmltypeout{latex2htmlLength vsize=\the\vsize}\lthtmltypeout{}%
\lthtmltypeout{latex2htmlLength hoffset=\the\hoffset}\lthtmltypeout{}%
\lthtmltypeout{latex2htmlLength voffset=\the\voffset}\lthtmltypeout{}%
\lthtmltypeout{latex2htmlLength topmargin=\the\topmargin}\lthtmltypeout{}%
\lthtmltypeout{latex2htmlLength topskip=\the\topskip}\lthtmltypeout{}%
\lthtmltypeout{latex2htmlLength headheight=\the\headheight}\lthtmltypeout{}%
\lthtmltypeout{latex2htmlLength headsep=\the\headsep}\lthtmltypeout{}%
\lthtmltypeout{latex2htmlLength parskip=\the\parskip}\lthtmltypeout{}%
\lthtmltypeout{latex2htmlLength oddsidemargin=\the\oddsidemargin}\lthtmltypeout{}%
\makeatletter
\if@twoside\lthtmltypeout{latex2htmlLength evensidemargin=\the\evensidemargin}%
\else\lthtmltypeout{latex2htmlLength evensidemargin=\the\oddsidemargin}\fi%
\lthtmltypeout{}%
\makeatother
\setcounter{page}{1}
\onecolumn

% !!! IMAGES START HERE !!!

\setcounter{pytexcount}{0}
\setcounter{pytexlinecountstart}{0}
\setcounter{pytexlinecountend}{0}
\stepcounter{chapter}
\stepcounter{section}
\stepcounter{section}
\stepcounter{section}
\stepcounter{section}
\stepcounter{chapter}
\stepcounter{chapter}
\stepcounter{section}
\addtocounter{pytexcount}{1}
\setcounter{pytexsubcount}{0}
{\newpage\clearpage
\lthtmlfigureA{lstlisting99}%
\begin{lstlisting}[language=Python]
# Python
from casadi import *
\end{lstlisting}%
\lthtmlfigureZ
\lthtmlcheckvsize\clearpage}

{\newpage\clearpage
\lthtmlfigureA{lstlisting104}%
\begin{lstlisting}[language=Matlab]
import casadi.*
\end{lstlisting}%
\lthtmlfigureZ
\lthtmlcheckvsize\clearpage}

\addtocounter{pytexsubcount}{1}
{\newpage\clearpage
\lthtmlfigureA{lstlisting112}%
\begin{lstlisting}[language=Python]
# Python
x = MX.sym('x')
\end{lstlisting}%
\lthtmlfigureZ
\lthtmlcheckvsize\clearpage}

{\newpage\clearpage
\lthtmlfigureA{lstlisting117}%
\begin{lstlisting}[language=Matlab]
x = MX.sym('x');
\end{lstlisting}%
\lthtmlfigureZ
\lthtmlcheckvsize\clearpage}

\addtocounter{pytexsubcount}{1}
{\newpage\clearpage
\lthtmlfigureA{lstlisting125}%
\begin{lstlisting}[language=Python]
# Python
y = SX.sym('y',5)
Z = SX.sym('Z',4,2)
\end{lstlisting}%
\lthtmlfigureZ
\lthtmlcheckvsize\clearpage}

{\newpage\clearpage
\lthtmlfigureA{lstlisting130}%
\begin{lstlisting}[language=Matlab]
y = SX.sym('y',5);
Z = SX.sym('Z',4,2);
\end{lstlisting}%
\lthtmlfigureZ
\lthtmlcheckvsize\clearpage}

\addtocounter{pytexsubcount}{1}
{\newpage\clearpage
\lthtmlfigureA{lstlisting139}%
\begin{lstlisting}[language=Python]
# Python
f = x**2 + 10
f = sqrt(f)
print 'f:', f
\end{lstlisting}%
\lthtmlfigureZ
\lthtmlcheckvsize\clearpage}

{\newpage\clearpage
\lthtmlfigureA{lstlisting144}%
\begin{lstlisting}[language=Matlab]
f = x^2 + 10;
f = sqrt(f);
display(f)
\end{lstlisting}%
\lthtmlfigureZ
\lthtmlcheckvsize\clearpage}

\addtocounter{pytexsubcount}{1}
{\newpage\clearpage
\lthtmlinlinemathA{tex2html_wrap_inline5154}%
$ n$%
\lthtmlinlinemathZ
\lthtmlcheckvsize\clearpage}

{\newpage\clearpage
\lthtmlinlinemathA{tex2html_wrap_inline5156}%
$ m$%
\lthtmlinlinemathZ
\lthtmlcheckvsize\clearpage}

\addtocounter{pytexsubcount}{1}
{\newpage\clearpage
\lthtmlinlinemathA{tex2html_wrap_inline5161}%
$ 00$%
\lthtmlinlinemathZ
\lthtmlcheckvsize\clearpage}

{\newpage\clearpage
\lthtmlfigureA{lstlisting162}%
\begin{lstlisting}[language=Python]
# Python
print 'B4:', B4
\end{lstlisting}%
\lthtmlfigureZ
\lthtmlcheckvsize\clearpage}

{\newpage\clearpage
\lthtmlfigureA{lstlisting167}%
\begin{lstlisting}[language=Matlab]
display(B4)
\end{lstlisting}%
\lthtmlfigureZ
\lthtmlcheckvsize\clearpage}

\addtocounter{pytexsubcount}{1}
{\newpage\clearpage
\lthtmlinlinemathA{tex2html_wrap_inline5186}%
$ v$%
\lthtmlinlinemathZ
\lthtmlcheckvsize\clearpage}

{\newpage\clearpage
\lthtmlfigureA{lstlisting202}%
\begin{lstlisting}[language=C++]
// C++
#include <casadi/casadi.hpp>
using namespace casadi;
int main() {
  SX x = SX::sym("x");
  SX y = SX::sym("y",5);
  SX Z = SX::sym("Z",4,2)
  SX f = pow(x,2) + 10;
  f = sqrt(f);
  std::cout << "f: " << f << std::endl;
  return 0;
}
\end{lstlisting}%
\lthtmlfigureZ
\lthtmlcheckvsize\clearpage}

\stepcounter{section}
\addtocounter{pytexsubcount}{1}
{\newpage\clearpage
\lthtmlfigureA{lstlisting223}%
\begin{lstlisting}[language=Matlab]
C = DM(2,3);
\par
C_dense = full(C);
\par
C_sparse = sparse(C);
\par
\end{lstlisting}%
\lthtmlfigureZ
\lthtmlcheckvsize\clearpage}

\stepcounter{section}
{\newpage\clearpage
\lthtmlfigureA{lstlisting234}%
\begin{lstlisting}[language=Python]
# Python
x = SX.sym('x',2,2)
y = SX.sym('y')
f = 3*x + y
print f
print f.shape
\end{lstlisting}%
\lthtmlfigureZ
\lthtmlcheckvsize\clearpage}

{\newpage\clearpage
\lthtmlfigureA{lstlisting239}%
\begin{lstlisting}[language=Matlab]
x = SX.sym('x',2,2);
y = SX.sym('y');
f = 3*x + y;
disp(f)
disp(size(f))
\end{lstlisting}%
\lthtmlfigureZ
\lthtmlcheckvsize\clearpage}

\addtocounter{pytexsubcount}{1}
{\newpage\clearpage
\lthtmlinlinemathA{tex2html_wrap_inline5203}%
$ \mathbb{R} \rightarrow \mathbb{R}$%
\lthtmlinlinemathZ
\lthtmlcheckvsize\clearpage}

{\newpage\clearpage
\lthtmlinlinemathA{tex2html_wrap_inline5205}%
$ \mathbb{R} \times \mathbb{R} \rightarrow \mathbb{R}$%
\lthtmlinlinemathZ
\lthtmlcheckvsize\clearpage}

{\newpage\clearpage
\lthtmlinlinemathA{tex2html_wrap_inline5207}%
$ \mathbb{R}^{n_1 \times m_1} \times \ldots \times \mathbb{R}^{n_N \times m_N} \rightarrow \mathbb{R}^{p_1 \times q_1} \times \ldots \times \mathbb{R}^{p_M \times q_M}$%
\lthtmlinlinemathZ
\lthtmlcheckvsize\clearpage}

{\newpage\clearpage
\lthtmlfigureA{lstlisting270}%
\begin{lstlisting}[language=Python]
# Python
x = MX.sym('x',2,2)
y = MX.sym('y')
f = 3*x + y
print f
print f.shape
\end{lstlisting}%
\lthtmlfigureZ
\lthtmlcheckvsize\clearpage}

{\newpage\clearpage
\lthtmlfigureA{lstlisting275}%
\begin{lstlisting}[language=Matlab]
x = MX.sym('x',2,2);
y = MX.sym('y');
f = 3*x + y;
disp(f)
disp(size(f))
\end{lstlisting}%
\lthtmlfigureZ
\lthtmlcheckvsize\clearpage}

\addtocounter{pytexsubcount}{1}
{\newpage\clearpage
\lthtmlfigureA{lstlisting288}%
\begin{lstlisting}[language=Python]
# Python
x = MX.sym('x',2,2)
print x[0,0]
\end{lstlisting}%
\lthtmlfigureZ
\lthtmlcheckvsize\clearpage}

{\newpage\clearpage
\lthtmlfigureA{lstlisting293}%
\begin{lstlisting}[language=Matlab]
x = MX.sym('x',2,2);
x(1,1)
\end{lstlisting}%
\lthtmlfigureZ
\lthtmlcheckvsize\clearpage}

\addtocounter{pytexsubcount}{1}
{\newpage\clearpage
\lthtmlfigureA{lstlisting303}%
\begin{lstlisting}[language=Python]
# Python
x = MX.sym('x',2)
A = MX(2,2)
A[0,0] = x[0]
A[1,1] = x[0]+x[1]
print 'A:', A
\end{lstlisting}%
\lthtmlfigureZ
\lthtmlcheckvsize\clearpage}

{\newpage\clearpage
\lthtmlfigureA{lstlisting308}%
\begin{lstlisting}[language=Matlab]
x = MX.sym('x',2);
A = MX(2,2);
A(1,1) = x(1);
A(2,2) = x(1)+x(2);
display(A)
\end{lstlisting}%
\lthtmlfigureZ
\lthtmlcheckvsize\clearpage}

\addtocounter{pytexsubcount}{1}
\stepcounter{section}
\stepcounter{section}
{\newpage\clearpage
\lthtmlfigureA{lstlisting352}%
\begin{lstlisting}[language=Python]
# Python
print SX.sym('x',Sparsity.lower(3))
\end{lstlisting}%
\lthtmlfigureZ
\lthtmlcheckvsize\clearpage}

{\newpage\clearpage
\lthtmlfigureA{lstlisting354}%
\begin{lstlisting}[language=Matlab]
disp(SX.sym('x',Sparsity.lower(3)))
\end{lstlisting}%
\lthtmlfigureZ
\lthtmlcheckvsize\clearpage}

\addtocounter{pytexsubcount}{1}
\stepcounter{subsection}
{\newpage\clearpage
\lthtmlfigureA{lstlisting369}%
\begin{lstlisting}[language=Python]
# Python
M = SX([[3,7],[4,5]])
print M[0,:]
M[0,:] = 1
print M
\end{lstlisting}%
\lthtmlfigureZ
\lthtmlcheckvsize\clearpage}

{\newpage\clearpage
\lthtmlfigureA{lstlisting374}%
\begin{lstlisting}[language=Matlab]
M = SX([3,7;4,5]);
disp(M(1,:))
M(1,:) = 1;
disp(M)
\end{lstlisting}%
\lthtmlfigureZ
\lthtmlcheckvsize\clearpage}

\addtocounter{pytexsubcount}{1}
{\newpage\clearpage
\lthtmlinlinemathA{tex2html_wrap_inline5242}%
$ M$%
\lthtmlinlinemathZ
\lthtmlcheckvsize\clearpage}

\addtocounter{pytexsubcount}{1}
\stepcounter{paragraph}
{\newpage\clearpage
\lthtmlfigureA{lstlisting384}%
\begin{lstlisting}[language=Python]
# Python
M = diag(SX([3,4,5,6]))
print M
\end{lstlisting}%
\lthtmlfigureZ
\lthtmlcheckvsize\clearpage}

{\newpage\clearpage
\lthtmlfigureA{lstlisting389}%
\begin{lstlisting}[language=Matlab]
M = diag(SX([3,4,5,6]));
disp(M)
\end{lstlisting}%
\lthtmlfigureZ
\lthtmlcheckvsize\clearpage}

\addtocounter{pytexsubcount}{1}
{\newpage\clearpage
\lthtmlfigureA{lstlisting396}%
\begin{lstlisting}[language=Python]
print M[0,0], M[1,0], M[-1,-1]
\end{lstlisting}%
\lthtmlfigureZ
\lthtmlcheckvsize\clearpage}

{\newpage\clearpage
\lthtmlfigureA{lstlisting401}%
\begin{lstlisting}[language=Matlab]
M(1,1), M(2,1), M(end,end)
\end{lstlisting}%
\lthtmlfigureZ
\lthtmlcheckvsize\clearpage}

\addtocounter{pytexsubcount}{1}
{\newpage\clearpage
\lthtmlfigureA{lstlisting408}%
\begin{lstlisting}[language=Python]
print M[5], M[-6]
\end{lstlisting}%
\lthtmlfigureZ
\lthtmlcheckvsize\clearpage}

{\newpage\clearpage
\lthtmlfigureA{lstlisting413}%
\begin{lstlisting}[language=Matlab]
M(6), M(end-5)
\end{lstlisting}%
\lthtmlfigureZ
\lthtmlcheckvsize\clearpage}

\addtocounter{pytexsubcount}{1}
\stepcounter{paragraph}
{\newpage\clearpage
\lthtmlfigureA{lstlisting424}%
\begin{lstlisting}[language=Python]
print M[:,1]
\end{lstlisting}%
\lthtmlfigureZ
\lthtmlcheckvsize\clearpage}

{\newpage\clearpage
\lthtmlfigureA{lstlisting429}%
\begin{lstlisting}[language=Matlab]
disp(M(:,2))
\end{lstlisting}%
\lthtmlfigureZ
\lthtmlcheckvsize\clearpage}

\addtocounter{pytexsubcount}{1}
{\newpage\clearpage
\lthtmlfigureA{lstlisting436}%
\begin{lstlisting}[language=Python]
print M[1:,1:4:2]
\end{lstlisting}%
\lthtmlfigureZ
\lthtmlcheckvsize\clearpage}

{\newpage\clearpage
\lthtmlfigureA{lstlisting441}%
\begin{lstlisting}[language=Matlab]
disp(M(2:end,2:2:4))
\end{lstlisting}%
\lthtmlfigureZ
\lthtmlcheckvsize\clearpage}

\addtocounter{pytexsubcount}{1}
\stepcounter{paragraph}
{\newpage\clearpage
\lthtmlfigureA{lstlisting452}%
\begin{lstlisting}[language=Python]
M = SX([[3,7,8,9],[4,5,6,1]])
print M
\end{lstlisting}%
\lthtmlfigureZ
\lthtmlcheckvsize\clearpage}

{\newpage\clearpage
\lthtmlfigureA{lstlisting457}%
\begin{lstlisting}[language=Matlab]
M = SX([3 7 8 9; 4 5 6 1]);
disp(M)
\end{lstlisting}%
\lthtmlfigureZ
\lthtmlcheckvsize\clearpage}

\addtocounter{pytexsubcount}{1}
{\newpage\clearpage
\lthtmlfigureA{lstlisting464}%
\begin{lstlisting}[language=Python]
print M[0,[0,3]], M[[5,-6]]
\end{lstlisting}%
\lthtmlfigureZ
\lthtmlcheckvsize\clearpage}

{\newpage\clearpage
\lthtmlfigureA{lstlisting469}%
\begin{lstlisting}[language=Matlab]
M(1,[1,4]), M([6,numel(M)-5])
\end{lstlisting}%
\lthtmlfigureZ
\lthtmlcheckvsize\clearpage}

\addtocounter{pytexsubcount}{1}
\stepcounter{section}
{\newpage\clearpage
\lthtmlfigureA{lstlisting477}%
\begin{lstlisting}[language=Python]
x = SX.sym('x')
y = SX.sym('y',2,2)
print sin(y)-x
\end{lstlisting}%
\lthtmlfigureZ
\lthtmlcheckvsize\clearpage}

{\newpage\clearpage
\lthtmlfigureA{lstlisting482}%
\begin{lstlisting}[language=Matlab]
x = SX.sym('x');
y = SX.sym('y',2,2);
sin(y)-x
\end{lstlisting}%
\lthtmlfigureZ
\lthtmlcheckvsize\clearpage}

\addtocounter{pytexsubcount}{1}
{\newpage\clearpage
\lthtmlfigureA{lstlisting491}%
\begin{lstlisting}[language=Python]
print y*y, mtimes(y,y)
\end{lstlisting}%
\lthtmlfigureZ
\lthtmlcheckvsize\clearpage}

{\newpage\clearpage
\lthtmlfigureA{lstlisting496}%
\begin{lstlisting}[language=Matlab]
y.*y, y*y
\end{lstlisting}%
\lthtmlfigureZ
\lthtmlcheckvsize\clearpage}

\addtocounter{pytexsubcount}{1}
{\newpage\clearpage
\lthtmlfigureA{lstlisting508}%
\begin{lstlisting}[language=Python]
print y.T
\end{lstlisting}%
\lthtmlfigureZ
\lthtmlcheckvsize\clearpage}

{\newpage\clearpage
\lthtmlfigureA{lstlisting513}%
\begin{lstlisting}[language=Matlab]
y'
\end{lstlisting}%
\lthtmlfigureZ
\lthtmlcheckvsize\clearpage}

\addtocounter{pytexsubcount}{1}
{\newpage\clearpage
\lthtmlfigureA{lstlisting521}%
\begin{lstlisting}[language=Python]
x = SX.eye(4)
print reshape(x,2,8)
\end{lstlisting}%
\lthtmlfigureZ
\lthtmlcheckvsize\clearpage}

{\newpage\clearpage
\lthtmlfigureA{lstlisting526}%
\begin{lstlisting}[language=Matlab]
x = SX.eye(4);
reshape(x,2,8)
\end{lstlisting}%
\lthtmlfigureZ
\lthtmlcheckvsize\clearpage}

\addtocounter{pytexsubcount}{1}
{\newpage\clearpage
\lthtmlfigureA{lstlisting536}%
\begin{lstlisting}[language=Python]
x = SX.sym('x',5)
y = SX.sym('y',5)
print vertcat(x,y)
\end{lstlisting}%
\lthtmlfigureZ
\lthtmlcheckvsize\clearpage}

{\newpage\clearpage
\lthtmlfigureA{lstlisting541}%
\begin{lstlisting}[language=Matlab]
x = SX.sym('x',5);
y = SX.sym('y',5);
[x;y]
\end{lstlisting}%
\lthtmlfigureZ
\lthtmlcheckvsize\clearpage}

\addtocounter{pytexsubcount}{1}
{\newpage\clearpage
\lthtmlfigureA{lstlisting548}%
\begin{lstlisting}[language=Python]
print horzcat(x,y)
\end{lstlisting}%
\lthtmlfigureZ
\lthtmlcheckvsize\clearpage}

{\newpage\clearpage
\lthtmlfigureA{lstlisting553}%
\begin{lstlisting}[language=Matlab]
[x,y]
\end{lstlisting}%
\lthtmlfigureZ
\lthtmlcheckvsize\clearpage}

\addtocounter{pytexsubcount}{1}
{\newpage\clearpage
\lthtmlinlinemathA{tex2html_wrap_inline5276}%
$ n+1$%
\lthtmlinlinemathZ
\lthtmlcheckvsize\clearpage}

{\newpage\clearpage
\lthtmlinlinemathA{tex2html_wrap_inline5278}%
$ i$%
\lthtmlinlinemathZ
\lthtmlcheckvsize\clearpage}

{\newpage\clearpage
\lthtmlinlinemathA{tex2html_wrap_inline5280}%
$ c$%
\lthtmlinlinemathZ
\lthtmlcheckvsize\clearpage}

{\newpage\clearpage
\lthtmlinlinemathA{tex2html_wrap_inline5282}%
$ \textit{offset}[i] \le c < \textit{offset}[i+1]$%
\lthtmlinlinemathZ
\lthtmlcheckvsize\clearpage}

{\newpage\clearpage
\lthtmlfigureA{lstlisting565}%
\begin{lstlisting}[language=Python]
x = SX.sym('x',5,2)
w = horzsplit(x,[0,1,2])
print w[0], w[1]
\end{lstlisting}%
\lthtmlfigureZ
\lthtmlcheckvsize\clearpage}

{\newpage\clearpage
\lthtmlfigureA{lstlisting570}%
\begin{lstlisting}[language=Matlab]
x = SX.sym('x',5,2);
w = horzsplit(x,[0,1,2]);
w{1}, w{2}
\end{lstlisting}%
\lthtmlfigureZ
\lthtmlcheckvsize\clearpage}

\addtocounter{pytexsubcount}{1}
{\newpage\clearpage
\lthtmlfigureA{lstlisting580}%
\begin{lstlisting}[language=Python]
w = vertsplit(x,[0,3,5])
print w[0], w[1]
\end{lstlisting}%
\lthtmlfigureZ
\lthtmlcheckvsize\clearpage}

{\newpage\clearpage
\lthtmlfigureA{lstlisting585}%
\begin{lstlisting}[language=Matlab]
w = vertsplit(x,[0,3,5]);
w{1}, w{2}
\end{lstlisting}%
\lthtmlfigureZ
\lthtmlcheckvsize\clearpage}

\addtocounter{pytexsubcount}{1}
{\newpage\clearpage
\lthtmlfigureA{lstlisting594}%
\begin{lstlisting}[language=Python]
w = [x[0:3,:], x[3:5,:]]
print w[0], w[1]
\end{lstlisting}%
\lthtmlfigureZ
\lthtmlcheckvsize\clearpage}

{\newpage\clearpage
\lthtmlfigureA{lstlisting599}%
\begin{lstlisting}[language=Matlab]
w = {x(1:3,:), x(4:5,:)};
w{1}, w{2}
\end{lstlisting}%
\lthtmlfigureZ
\lthtmlcheckvsize\clearpage}

\addtocounter{pytexsubcount}{1}
{\newpage\clearpage
\lthtmlinlinemathA{tex2html_wrap_inline5290}%
$ <A,B> :=$%
\lthtmlinlinemathZ
\lthtmlcheckvsize\clearpage}

{\newpage\clearpage
\lthtmlinlinemathA{tex2html_wrap_inline5291}%
$ (A \, B) = \sum_{i,j} \, A_{i,j} \, B_{i,j}$%
\lthtmlinlinemathZ
\lthtmlcheckvsize\clearpage}

{\newpage\clearpage
\lthtmlfigureA{lstlisting619}%
\begin{lstlisting}[language=Python]
x = SX.sym('x',2,2)
print dot(x,x)
\end{lstlisting}%
\lthtmlfigureZ
\lthtmlcheckvsize\clearpage}

{\newpage\clearpage
\lthtmlfigureA{lstlisting624}%
\begin{lstlisting}[language=Matlab]
x = SX.sym('x',2,2)
dot(x,x)
\end{lstlisting}%
\lthtmlfigureZ
\lthtmlcheckvsize\clearpage}

\addtocounter{pytexsubcount}{1}
\stepcounter{section}
{\newpage\clearpage
\lthtmlfigureA{lstlisting638}%
\begin{lstlisting}[language=Python]
y = SX.sym('y',10,1)
print y.shape
\end{lstlisting}%
\lthtmlfigureZ
\lthtmlcheckvsize\clearpage}

{\newpage\clearpage
\lthtmlfigureA{lstlisting643}%
\begin{lstlisting}[language=Matlab]
y = SX.sym('y',10,1);
size(y)
\end{lstlisting}%
\lthtmlfigureZ
\lthtmlcheckvsize\clearpage}

\addtocounter{pytexsubcount}{1}
{\newpage\clearpage
\lthtmlinlinemathA{tex2html_wrap_inline5311}%
$ \textit{nrow} * \textit{ncol}$%
\lthtmlinlinemathZ
\lthtmlcheckvsize\clearpage}

\stepcounter{section}
{\newpage\clearpage
\lthtmlfigureA{lstlisting682}%
\begin{lstlisting}[language=Python]
A = MX.sym('A',3,3)
b = MX.sym('b',3)
print solve(A,b)
\end{lstlisting}%
\lthtmlfigureZ
\lthtmlcheckvsize\clearpage}

{\newpage\clearpage
\lthtmlfigureA{lstlisting687}%
\begin{lstlisting}[language=Matlab]
A = MX.sym('A',3,3);
b = MX.sym('b',3);
solve(A,b)
\end{lstlisting}%
\lthtmlfigureZ
\lthtmlcheckvsize\clearpage}

\addtocounter{pytexsubcount}{1}
\stepcounter{section}
{\newpage\clearpage
\lthtmlinlinemathA{tex2html_wrap_inline5319}%
$ f: \mathbb{R}^N \rightarrow \mathbb{R}^M$%
\lthtmlinlinemathZ
\lthtmlcheckvsize\clearpage}

{\newpage\clearpage
\lthtmlinlinemathA{tex2html_wrap_indisplay5321}%
$\displaystyle y = f(x),$%
\lthtmlindisplaymathZ
\lthtmlcheckvsize\clearpage}

{\newpage\clearpage
\lthtmlinlinemathA{tex2html_wrap_indisplay5323}%
$\displaystyle \hat{y} = \frac{\partial f}{\partial x} \, \hat{x}.$%
\lthtmlindisplaymathZ
\lthtmlcheckvsize\clearpage}

{\newpage\clearpage
\lthtmlinlinemathA{tex2html_wrap_indisplay5325}%
$\displaystyle \bar{x} = \left(\frac{\partial f}{\partial x}\right)^{\text{T}} \, \bar{y}.$%
\lthtmlindisplaymathZ
\lthtmlcheckvsize\clearpage}

{\newpage\clearpage
\lthtmlinlinemathA{tex2html_wrap_inline5327}%
$ f(x)$%
\lthtmlinlinemathZ
\lthtmlcheckvsize\clearpage}

{\newpage\clearpage
\lthtmlinlinemathA{tex2html_wrap_inline5329}%
$ x$%
\lthtmlinlinemathZ
\lthtmlcheckvsize\clearpage}

{\newpage\clearpage
\lthtmlfigureA{lstlisting720}%
\begin{lstlisting}[language=Python]
A = SX.sym('A',3,2)
x = SX.sym('x',2)
print jacobian(mtimes(A,x),x)
\end{lstlisting}%
\lthtmlfigureZ
\lthtmlcheckvsize\clearpage}

{\newpage\clearpage
\lthtmlfigureA{lstlisting725}%
\begin{lstlisting}[language=Matlab]
A = SX.sym('A',3,2);
x = SX.sym('x',2);
jacobian(A*x,x)
\end{lstlisting}%
\lthtmlfigureZ
\lthtmlcheckvsize\clearpage}

\addtocounter{pytexsubcount}{1}
{\newpage\clearpage
\lthtmlfigureA{lstlisting732}%
\begin{lstlisting}[language=Python]
print gradient(dot(A,A),A)
\end{lstlisting}%
\lthtmlfigureZ
\lthtmlcheckvsize\clearpage}

{\newpage\clearpage
\lthtmlfigureA{lstlisting737}%
\begin{lstlisting}[language=Matlab]
gradient(dot(A,A),A)
\end{lstlisting}%
\lthtmlfigureZ
\lthtmlcheckvsize\clearpage}

\addtocounter{pytexsubcount}{1}
{\newpage\clearpage
\lthtmlfigureA{lstlisting744}%
\begin{lstlisting}[language=Python]
[H,g] = hessian(dot(x,x),x)
print 'H:', H
\end{lstlisting}%
\lthtmlfigureZ
\lthtmlcheckvsize\clearpage}

{\newpage\clearpage
\lthtmlfigureA{lstlisting749}%
\begin{lstlisting}[language=Matlab]
[H,g] = hessian(dot(x,x),x);
display(H)
\end{lstlisting}%
\lthtmlfigureZ
\lthtmlcheckvsize\clearpage}

\addtocounter{pytexsubcount}{1}
{\newpage\clearpage
\lthtmlfigureA{lstlisting757}%
\begin{lstlisting}[language=Python]
v = SX.sym('v',2)
f = mtimes(A,x)
print jtimes(f,x,v)
\end{lstlisting}%
\lthtmlfigureZ
\lthtmlcheckvsize\clearpage}

{\newpage\clearpage
\lthtmlfigureA{lstlisting762}%
\begin{lstlisting}[language=Matlab]
v = SX.sym('v',2);
f = A*x;
jtimes(f,x,v)
\end{lstlisting}%
\lthtmlfigureZ
\lthtmlcheckvsize\clearpage}

\addtocounter{pytexsubcount}{1}
{\newpage\clearpage
\lthtmlfigureA{lstlisting770}%
\begin{lstlisting}[language=Python]
w = SX.sym('w',3)
f = mtimes(A,x)
print jtimes(f,x,w,True)
\end{lstlisting}%
\lthtmlfigureZ
\lthtmlcheckvsize\clearpage}

{\newpage\clearpage
\lthtmlfigureA{lstlisting775}%
\begin{lstlisting}[language=Matlab]
w = SX.sym('w',3);
f = A*x
jtimes(f,x,w,true)
\end{lstlisting}%
\lthtmlfigureZ
\lthtmlcheckvsize\clearpage}

\addtocounter{pytexsubcount}{1}
\stepcounter{chapter}
{\newpage\clearpage
\lthtmlfigureA{lstlisting783}%
\begin{lstlisting}[language=Python]
f = functionname(name, arguments, ..., [options])
\end{lstlisting}%
\lthtmlfigureZ
\lthtmlcheckvsize\clearpage}

{\newpage\clearpage
\lthtmlfigureA{lstlisting789}%
\begin{lstlisting}[language=Python]
# Python
x = SX.sym('x',2)
y = SX.sym('y')
f = Function('f',[x,y],\
           [x,sin(y)*x])
\end{lstlisting}%
\lthtmlfigureZ
\lthtmlcheckvsize\clearpage}

{\newpage\clearpage
\lthtmlfigureA{lstlisting794}%
\begin{lstlisting}[language=Matlab]
x = SX.sym('x',2);
y = SX.sym('y');
f = Function('f',{x,y},...
           {x,sin(y)*x});
\end{lstlisting}%
\lthtmlfigureZ
\lthtmlcheckvsize\clearpage}

\addtocounter{pytexsubcount}{1}
{\newpage\clearpage
\lthtmlinlinemathA{tex2html_wrap_inline5345}%
$ f : \mathbb{R}^{2} \times \mathbb{R} \rightarrow \mathbb{R}^{2} \times \mathbb{R}^{2}, \quad (x,y) \mapsto (x,\sin(y) x)$%
\lthtmlinlinemathZ
\lthtmlcheckvsize\clearpage}

{\newpage\clearpage
\lthtmlfigureA{lstlisting811}%
\begin{lstlisting}[language=Python]
# Python
x = MX.sym('x',2)
y = MX.sym('y')
f = Function('f',[x,y],\
             [x,sin(y)*x])
\end{lstlisting}%
\lthtmlfigureZ
\lthtmlcheckvsize\clearpage}

{\newpage\clearpage
\lthtmlfigureA{lstlisting816}%
\begin{lstlisting}[language=Matlab]
x = MX.sym('x',2);
y = MX.sym('y');
f = Function('f',{x,y},...
             {x,sin(y)*x});
\end{lstlisting}%
\lthtmlfigureZ
\lthtmlcheckvsize\clearpage}

\addtocounter{pytexsubcount}{1}
{\newpage\clearpage
\lthtmlfigureA{lstlisting827}%
\begin{lstlisting}[language=Python]
# Python
x = MX.sym('x',2)
y = MX.sym('y')
f = Function('f',[x,y],\
      [x,sin(y)*x],\
      ['x','y'],['r','q'])
\end{lstlisting}%
\lthtmlfigureZ
\lthtmlcheckvsize\clearpage}

{\newpage\clearpage
\lthtmlfigureA{lstlisting832}%
\begin{lstlisting}[language=Matlab]
x = MX.sym('x',2);
y = MX.sym('y');
f = Function('f',{x,y},...
      {x,sin(y)*x},...
      {'x','y'},{'r','q'});
\end{lstlisting}%
\lthtmlfigureZ
\lthtmlcheckvsize\clearpage}

\addtocounter{pytexsubcount}{1}
\stepcounter{section}
{\newpage\clearpage
\lthtmlfigureA{lstlisting852}%
\begin{lstlisting}[language=Python]
# Python
r0, q0 = f(1.1,3.3)
print 'r0:',r0
print 'q0:',q0
\end{lstlisting}%
\lthtmlfigureZ
\lthtmlcheckvsize\clearpage}

{\newpage\clearpage
\lthtmlfigureA{lstlisting857}%
\begin{lstlisting}[language=Matlab]
[r0, q0] = f(1.1,3.3);
display(r0)
display(q0)
\end{lstlisting}%
\lthtmlfigureZ
\lthtmlcheckvsize\clearpage}

\addtocounter{pytexsubcount}{1}
{\newpage\clearpage
\lthtmlfigureA{lstlisting867}%
\begin{lstlisting}[language=Python]
# Python
res = f(x=1.1, y=3.3)
print 'res:', res
\end{lstlisting}%
\lthtmlfigureZ
\lthtmlcheckvsize\clearpage}

{\newpage\clearpage
\lthtmlfigureA{lstlisting872}%
\begin{lstlisting}[language=Matlab]
res = f('x',1.1,'y',3.3);
display(res)
\end{lstlisting}%
\lthtmlfigureZ
\lthtmlcheckvsize\clearpage}

\addtocounter{pytexsubcount}{1}
{\newpage\clearpage
\lthtmlfigureA{lstlisting882}%
\begin{lstlisting}[language=Python]
# Python
arg = [1.1,3.3]
res = f.call(arg)
print 'res:', res
arg = {'x':1.1,'y':3.3}
res = f.call(arg)
print 'res:', res
\end{lstlisting}%
\lthtmlfigureZ
\lthtmlcheckvsize\clearpage}

{\newpage\clearpage
\lthtmlfigureA{lstlisting888}%
\begin{lstlisting}[language=Matlab]
arg = {1.1,3.3};
res = f.call(arg);
display(res)
arg = struct('x',1.1,'y',3.3);
res = f.call(arg);
display(res)
\end{lstlisting}%
\lthtmlfigureZ
\lthtmlcheckvsize\clearpage}

\addtocounter{pytexsubcount}{1}
\stepcounter{section}
{\newpage\clearpage
\lthtmlfigureA{lstlisting900}%
\begin{lstlisting}[language=Python]
sx_function = mx_function.expand()
\end{lstlisting}%
\lthtmlfigureZ
\lthtmlcheckvsize\clearpage}

\stepcounter{section}
{\newpage\clearpage
\lthtmldisplayA{displaymath5360}%
\begin{equation*}\begin{aligned} &g_0(z, x_1, x_2, \ldots, x_n) &&= 0 \\&g_1(z, x_1, x_2, \ldots, x_n) &&= y_1 \\&g_2(z, x_1, x_2, \ldots, x_n) &&= y_2 \\&\qquad \vdots \qquad &&\qquad \\&g_m(z, x_1, x_2, \ldots, x_n) &&= y_m, \end{aligned}\end{equation*}%
\lthtmldisplayZ
\lthtmlcheckvsize\clearpage}

{\newpage\clearpage
\lthtmlinlinemathA{tex2html_wrap_inline5362}%
$ z$%
\lthtmlinlinemathZ
\lthtmlcheckvsize\clearpage}

{\newpage\clearpage
\lthtmlinlinemathA{tex2html_wrap_inline5364}%
$ x_1$%
\lthtmlinlinemathZ
\lthtmlcheckvsize\clearpage}

{\newpage\clearpage
\lthtmlinlinemathA{tex2html_wrap_inline5366}%
$ x_n$%
\lthtmlinlinemathZ
\lthtmlcheckvsize\clearpage}

{\newpage\clearpage
\lthtmlinlinemathA{tex2html_wrap_inline5368}%
$ y_1$%
\lthtmlinlinemathZ
\lthtmlcheckvsize\clearpage}

{\newpage\clearpage
\lthtmlinlinemathA{tex2html_wrap_inline5370}%
$ y_m$%
\lthtmlinlinemathZ
\lthtmlcheckvsize\clearpage}

{\newpage\clearpage
\lthtmlinlinemathA{tex2html_wrap_inline5372}%
$ g$%
\lthtmlinlinemathZ
\lthtmlcheckvsize\clearpage}

{\newpage\clearpage
\lthtmlinlinemathA{tex2html_wrap_inline5374}%
$ g_0$%
\lthtmlinlinemathZ
\lthtmlcheckvsize\clearpage}

{\newpage\clearpage
\lthtmlinlinemathA{tex2html_wrap_inline5376}%
$ g_m$%
\lthtmlinlinemathZ
\lthtmlcheckvsize\clearpage}

{\newpage\clearpage
\lthtmlinlinemathA{tex2html_wrap_inline5378}%
$ G: \{z_{\text{guess}}, x_1, x_2, \ldots, x_n\} \rightarrow \{z, y_1, y_2, \ldots, y_m\}$%
\lthtmlinlinemathZ
\lthtmlcheckvsize\clearpage}

{\newpage\clearpage
\lthtmlinlinemathA{tex2html_wrap_inline5382}%
$ n=m=1$%
\lthtmlinlinemathZ
\lthtmlcheckvsize\clearpage}

{\newpage\clearpage
\lthtmlfigureA{lstlisting913}%
\begin{lstlisting}[language=Python]
# Python
z = SX.sym('x',nz)
x = SX.sym('x',nx)
g0 = (an expression of x, z)
g1 = (an expression of x, z)
g = Function('g',[z,x],[g0,g1])
G = rootfinder('G','newton',g)
\end{lstlisting}%
\lthtmlfigureZ
\lthtmlcheckvsize\clearpage}

{\newpage\clearpage
\lthtmlfigureA{lstlisting918}%
\begin{lstlisting}[language=Matlab]
z = SX.sym('x',nz);
x = SX.sym('x',nx);
g0 = (an expression of x, z)
g1 = (an expression of x, z)
g = Function('g',{z,x},{g0,g1});
G = rootfinder('G','newton',g);
\end{lstlisting}%
\lthtmlfigureZ
\lthtmlcheckvsize\clearpage}

\stepcounter{section}
{\newpage\clearpage
\setcounter{equation}{1}
\lthtmldisplayA{subequations5387}%
\begin{subequations}\begin{align}  \dot{x} &= f_{\text{ode}}(t,x,z,p), \qquad x(0) = x_0 \\0  &= f_{\text{alg}}(t,x,z,p) \\\dot{q} &= f_{\text{quad}}(t,x,z,p), \qquad q(0) = 0 \end{align}\end{subequations}%
\lthtmldisplayZ
\lthtmlcheckvsize\clearpage}

\stepcounter{subsection}
{\newpage\clearpage
\setcounter{equation}{2}
\lthtmldisplayA{subequations5392}%
\setcounter{equation}{1}
\begin{subequations}\begin{align}  \dot{x} &= z+p, \\0  &= z \, \cos(z)-x \end{align}\end{subequations}%
\lthtmldisplayZ
\lthtmlcheckvsize\clearpage}

{\newpage\clearpage
\lthtmlfigureA{lstlisting952}%
\begin{lstlisting}[language=Python]
# Python
x = SX.sym('x'); z = SX.sym('z'); p = SX.sym('p')
dae = {'x':x, 'z':z, 'p':p, 'ode':z+p, 'alg':z*cos(z)-x}
F = integrator('F', 'idas', dae)
\end{lstlisting}%
\lthtmlfigureZ
\lthtmlcheckvsize\clearpage}

{\newpage\clearpage
\lthtmlfigureA{lstlisting955}%
\begin{lstlisting}[language=Matlab]
x = SX.sym('x'); z = SX.sym('z'); p = SX.sym('p');
dae = struct('x',x,'z',z,'p',p,'ode',z+p,'alg',z*cos(z)-x);
F = integrator('F', 'idas', dae);
\end{lstlisting}%
\lthtmlfigureZ
\lthtmlcheckvsize\clearpage}

\addtocounter{pytexsubcount}{1}
{\newpage\clearpage
\lthtmlinlinemathA{tex2html_wrap_inline5394}%
$ x(0)=0$%
\lthtmlinlinemathZ
\lthtmlcheckvsize\clearpage}

{\newpage\clearpage
\lthtmlinlinemathA{tex2html_wrap_inline5396}%
$ p=0.1$%
\lthtmlinlinemathZ
\lthtmlcheckvsize\clearpage}

{\newpage\clearpage
\lthtmlinlinemathA{tex2html_wrap_inline5398}%
$ z(0)=0$%
\lthtmlinlinemathZ
\lthtmlcheckvsize\clearpage}

{\newpage\clearpage
\lthtmlfigureA{lstlisting962}%
\begin{lstlisting}[language=Python]
# Python
r = F(x0=0, z0=0, p=0.1)
print r['xf']
\end{lstlisting}%
\lthtmlfigureZ
\lthtmlcheckvsize\clearpage}

{\newpage\clearpage
\lthtmlfigureA{lstlisting967}%
\begin{lstlisting}[language=Matlab]
r = F('x0',0,'z0',0,'p',0.1);
disp(r.xf)
\end{lstlisting}%
\lthtmlfigureZ
\lthtmlcheckvsize\clearpage}

\addtocounter{pytexsubcount}{1}
{\newpage\clearpage
\lthtmlinlinemathA{tex2html_wrap_inline5404}%
$ t$%
\lthtmlinlinemathZ
\lthtmlcheckvsize\clearpage}

\stepcounter{subsection}
\stepcounter{section}
{\newpage\clearpage
\lthtmldisplayA{displaymath5408}%
\begin{displaymath}\begin{array}{cc} \begin{array}{c} \text{minimize:} \\x \end{array} & f(x,p) \\\begin{array}{c} \text{subject to:} \end{array} & \begin{array}{rcl}   x_{\text{lb}} \le &  x   & \le x_{\text{ub}} \\g_{\text{lb}} \le &g(x,p)& \le g_{\text{ub}} \end{array} \end{array}\end{displaymath}%
\lthtmldisplayZ
\lthtmlcheckvsize\clearpage}

{\newpage\clearpage
\lthtmlinlinemathA{tex2html_wrap_inline5410}%
$ x \in \mathbb{R}^{nx}$%
\lthtmlinlinemathZ
\lthtmlcheckvsize\clearpage}

{\newpage\clearpage
\lthtmlinlinemathA{tex2html_wrap_inline5412}%
$ p \in \mathbb{R}^{np}$%
\lthtmlinlinemathZ
\lthtmlcheckvsize\clearpage}

{\newpage\clearpage
\lthtmlinlinemathA{tex2html_wrap_inline5414}%
$ L(x,\lambda) = f(x) + \lambda^{\text{T}} \, g(x))$%
\lthtmlinlinemathZ
\lthtmlcheckvsize\clearpage}

\stepcounter{subsection}
{\newpage\clearpage
\lthtmldisplayA{displaymath5419}%
\begin{displaymath}\begin{array}{cc} \begin{array}{c} \text{minimize:} \\x,y,z \end{array} & x^2 + 100 \, z^2  \\\begin{array}{c} \text{subject to:} \end{array} &  z+(1-x)^2-y = 0 \end{array}\end{displaymath}%
\lthtmldisplayZ
\lthtmlcheckvsize\clearpage}

{\newpage\clearpage
\lthtmlfigureA{lstlisting1018}%
\begin{lstlisting}[language=Python]
# Python
x = SX.sym('x'); y = SX.sym('y'); z = SX.sym('z')
nlp = {'x':vertcat(x,y,z), 'f':x**2+100*z**2, 'g':z+(1-x)**2-y}
S = nlpsol('S', 'ipopt', nlp)
\end{lstlisting}%
\lthtmlfigureZ
\lthtmlcheckvsize\clearpage}

{\newpage\clearpage
\lthtmlfigureA{lstlisting1021}%
\begin{lstlisting}[language=Matlab]
x = SX.sym('x'); y = SX.sym('y'); z = SX.sym('z');
nlp = struct('x',[x;y;z], 'f':x^2+100*z^2, 'g',z+(1-x)^2-y)
S = nlpsol('S', 'ipopt', nlp)
\end{lstlisting}%
\lthtmlfigureZ
\lthtmlcheckvsize\clearpage}

\addtocounter{pytexsubcount}{1}
{\newpage\clearpage
\lthtmlinlinemathA{tex2html_wrap_inline5421}%
$ [2.5,3.0,0.75]$%
\lthtmlinlinemathZ
\lthtmlcheckvsize\clearpage}

{\newpage\clearpage
\lthtmlfigureA{lstlisting1029}%
\begin{lstlisting}[language=Python]
# Python
r = S(x0=[2.5,3.0,0.75],\
      lbg=0, ubg=0)
x_opt = r['x']
print 'x_opt: ', x_opt
\end{lstlisting}%
\lthtmlfigureZ
\lthtmlcheckvsize\clearpage}

{\newpage\clearpage
\lthtmlfigureA{lstlisting1034}%
\begin{lstlisting}[language=Matlab]
r = S('x0',[2.5,3.0,0.75],...
      'lbg',0,'ubg',0);
x_opt = r.x;
display(x_opt)
\end{lstlisting}%
\lthtmlfigureZ
\lthtmlcheckvsize\clearpage}

\addtocounter{pytexsubcount}{1}
\addtocounter{pytexsubcount}{1}
\stepcounter{section}
\stepcounter{subsection}
{\newpage\clearpage
\lthtmlinlinemathA{tex2html_wrap_inline5427}%
$ f(x,p)$%
\lthtmlinlinemathZ
\lthtmlcheckvsize\clearpage}

{\newpage\clearpage
\lthtmlinlinemathA{tex2html_wrap_inline5431}%
$ g(x,p)$%
\lthtmlinlinemathZ
\lthtmlcheckvsize\clearpage}

{\newpage\clearpage
\lthtmldisplayA{displaymath5437}%
\begin{displaymath}\begin{array}{cc} \begin{array}{c} \text{minimize:} \\x,y \end{array} & x^2 + y^2  \\\begin{array}{c} \text{subject to:} \end{array} & x+y-10 \ge 0 \end{array}\end{displaymath}%
\lthtmldisplayZ
\lthtmlcheckvsize\clearpage}

{\newpage\clearpage
\lthtmlfigureA{lstlisting1061}%
\begin{lstlisting}[language=Python]
# Python
x = SX.sym('x'); y = SX.sym('y')
qp = {'x':vertcat(x,y), 'f':x**2+y**2, 'g':x+y-10}
S = qpsol('S', 'qpoases', qp)
\end{lstlisting}%
\lthtmlfigureZ
\lthtmlcheckvsize\clearpage}

{\newpage\clearpage
\lthtmlfigureA{lstlisting1064}%
\begin{lstlisting}[language=Matlab]
x = SX.sym('x'); y = SX.sym('y')
qp = struct('x',[x;y], 'f':x^2+y^2, 'g',x+y-10)
S = qpsol('S', 'qpoases', qp)
\end{lstlisting}%
\lthtmlfigureZ
\lthtmlcheckvsize\clearpage}

\addtocounter{pytexsubcount}{1}
{\newpage\clearpage
\lthtmlfigureA{lstlisting1073}%
\begin{lstlisting}[language=Python]
# Python
r = S(lbg=0)
x_opt = r['x']
print 'x_opt: ', x_opt
\end{lstlisting}%
\lthtmlfigureZ
\lthtmlcheckvsize\clearpage}

{\newpage\clearpage
\lthtmlfigureA{lstlisting1078}%
\begin{lstlisting}[language=Matlab]
r = S('lbg',0);
x_opt = r.x;
display(x_opt)
\end{lstlisting}%
\lthtmlfigureZ
\lthtmlcheckvsize\clearpage}

\addtocounter{pytexsubcount}{1}
\addtocounter{pytexsubcount}{1}
\stepcounter{subsection}
{\newpage\clearpage
\lthtmldisplayA{displaymath5442}%
\begin{displaymath}\begin{array}{cc} \begin{array}{c} \text{minimize:} \\x \end{array} & \frac{1}{2} x^\text{T}\, H \, x + g^\text{T}\, x \\\begin{array}{c} \text{subject to:} \end{array} & \begin{array}{rcl}   x_{\text{lb}} \le &  x   & \le x_{\text{ub}} \\a_{\text{lb}} \le & A \, x& \le a_{\text{ub}} \end{array} \end{array}\end{displaymath}%
\lthtmldisplayZ
\lthtmlcheckvsize\clearpage}

{\newpage\clearpage
\lthtmlfigureA{lstlisting1112}%
\begin{lstlisting}[language=Python]
# Python
H = 2*DM.eye(2)
A = DM.ones(1,2)
g = DM.zeros(2)
lba = 10.
\end{lstlisting}%
\lthtmlfigureZ
\lthtmlcheckvsize\clearpage}

{\newpage\clearpage
\lthtmlfigureA{lstlisting1117}%
\begin{lstlisting}[language=Matlab]
H = 2*DM.eye(2);
A = DM.ones(1,2);
g = DM.zeros(2);
lba = 10;
\end{lstlisting}%
\lthtmlfigureZ
\lthtmlcheckvsize\clearpage}

\addtocounter{pytexsubcount}{1}
{\newpage\clearpage
\lthtmlinlinemathA{tex2html_wrap_inline5446}%
$ H$%
\lthtmlinlinemathZ
\lthtmlcheckvsize\clearpage}

{\newpage\clearpage
\lthtmlinlinemathA{tex2html_wrap_inline5448}%
$ A$%
\lthtmlinlinemathZ
\lthtmlcheckvsize\clearpage}

{\newpage\clearpage
\lthtmlfigureA{lstlisting1125}%
\begin{lstlisting}[language=Python]
# Python
qp = {}
qp['h'] = H.sparsity()
qp['a'] = A.sparsity()
S = qpsol('S','qpoases',qp)
\end{lstlisting}%
\lthtmlfigureZ
\lthtmlcheckvsize\clearpage}

{\newpage\clearpage
\lthtmlfigureA{lstlisting1131}%
\begin{lstlisting}[language=Matlab]
qp = struct;
qp.h = H.sparsity();
qp.a = A.sparsity();
S = qpsol('S','qpoases',qp);
\end{lstlisting}%
\lthtmlfigureZ
\lthtmlcheckvsize\clearpage}

\addtocounter{pytexsubcount}{1}
{\newpage\clearpage
\lthtmlfigureA{lstlisting1141}%
\begin{lstlisting}[language=Python]
# Python
r = S(h=H, g=g, \
      a=A, lba=lba)
x_opt = r['x']
print 'x_opt: ', x_opt
\end{lstlisting}%
\lthtmlfigureZ
\lthtmlcheckvsize\clearpage}

{\newpage\clearpage
\lthtmlfigureA{lstlisting1146}%
\begin{lstlisting}[language=Matlab]
r = S('h', H, 'g', g,...
      'a', A, 'lba', lba);
x_opt = r.x;
display(x_opt)
\end{lstlisting}%
\lthtmlfigureZ
\lthtmlcheckvsize\clearpage}

\addtocounter{pytexsubcount}{1}
\addtocounter{pytexsubcount}{1}
\stepcounter{chapter}
\stepcounter{section}
{\newpage\clearpage
\lthtmlfigureA{lstlisting1164}%
\begin{lstlisting}[language=Python]
# Python
x = MX.sym('x',2)
y = MX.sym('y')
f = Function('f',[x,y],\
      [x,sin(y)*x],\
      ['x','y'],['r','q'])
f.generate('gen.c')
\end{lstlisting}%
\lthtmlfigureZ
\lthtmlcheckvsize\clearpage}

{\newpage\clearpage
\lthtmlfigureA{lstlisting1169}%
\begin{lstlisting}[language=Matlab]
x = MX.sym('x',2);
y = MX.sym('y');
f = Function('f',{x,y},...
      {x,sin(y)*x},...
      {'x','y'},{'r','q'});
f.generate('gen.c');
\end{lstlisting}%
\lthtmlfigureZ
\lthtmlcheckvsize\clearpage}

\addtocounter{pytexsubcount}{1}
{\newpage\clearpage
\lthtmlfigureA{lstlisting1185}%
\begin{lstlisting}[language=Python]
# Python
f = Function('f',[x],[sin(x)])
g = Function('g',[x],[cos(x)])
C = CodeGenerator()
C.add(f)
C.add(g)
C.generate('gen.c')
\end{lstlisting}%
\lthtmlfigureZ
\lthtmlcheckvsize\clearpage}

{\newpage\clearpage
\lthtmlfigureA{lstlisting1190}%
\begin{lstlisting}[language=Matlab]
f = Function('f',{x},{sin(x)});
g = Function('g',{x},{cos(x)});
C = CodeGenerator();
C.add(f);
C.add(g);
C.generate('gen.c');
\end{lstlisting}%
\lthtmlfigureZ
\lthtmlcheckvsize\clearpage}

\addtocounter{pytexsubcount}{1}
{\newpage\clearpage
\lthtmlfigureA{lstlisting1205}%
\begin{lstlisting}[language=Python]
# Python
f = Function('f',[x],[sin(x)])
opts = dict(main=True, \
            mex=True)
f.generate('ff.c',opts)
\end{lstlisting}%
\lthtmlfigureZ
\lthtmlcheckvsize\clearpage}

{\newpage\clearpage
\lthtmlfigureA{lstlisting1210}%
\begin{lstlisting}[language=Matlab]
f = Function('f',{x},{sin(x)});
opts = struct('main', true,...
              'mex', true);
f.generate('ff.c',opts);
\end{lstlisting}%
\lthtmlfigureZ
\lthtmlcheckvsize\clearpage}

\addtocounter{pytexsubcount}{1}
\stepcounter{section}
{\newpage\clearpage
\lthtmlfigureA{lstlisting1233}%
\begin{lstlisting}[language=sh]
gcc -fPIC -shared gen.c -o gen.so
\end{lstlisting}%
\lthtmlfigureZ
\lthtmlcheckvsize\clearpage}

{\newpage\clearpage
\lthtmlfigureA{lstlisting1241}%
\begin{lstlisting}[language=Python]
# Python
f = external('f', './ff.so')
print f(3.14)
\end{lstlisting}%
\lthtmlfigureZ
\lthtmlcheckvsize\clearpage}

{\newpage\clearpage
\lthtmlfigureA{lstlisting1246}%
\begin{lstlisting}[language=Matlab]
f = external('f', './ff.so');
disp(f(3.14))
\end{lstlisting}%
\lthtmlfigureZ
\lthtmlcheckvsize\clearpage}

\addtocounter{pytexsubcount}{1}
{\newpage\clearpage
\lthtmlfigureA{lstlisting1256}%
\begin{lstlisting}[language=Python]
# Python
C = Compiler('ff.c','clang')
f = external('f',C);
print f(3.14)
\end{lstlisting}%
\lthtmlfigureZ
\lthtmlcheckvsize\clearpage}

{\newpage\clearpage
\lthtmlfigureA{lstlisting1261}%
\begin{lstlisting}[language=Matlab]
C = Compiler('ff.c','clang');
f = external('f',C);
disp(f(3.14))
\end{lstlisting}%
\lthtmlfigureZ
\lthtmlcheckvsize\clearpage}

\addtocounter{pytexsubcount}{1}
{\newpage\clearpage
\lthtmlfigureA{lstlisting1270}%
\begin{lstlisting}[language=Matlab]
mex ff.c -largeArrayDims
disp(ff('f', 3.14))
\end{lstlisting}%
\lthtmlfigureZ
\lthtmlcheckvsize\clearpage}

\addtocounter{pytexsubcount}{1}
{\newpage\clearpage
\lthtmlfigureA{lstlisting1278}%
\begin{lstlisting}[language=sh]
# Command line
echo 3.14 3.14 > ff_in.txt
gcc ff.c -o ff
./ff f < ff_in.txt > ff_out.txt
cat ff_out.txt
\end{lstlisting}%
\lthtmlfigureZ
\lthtmlcheckvsize\clearpage}

\addtocounter{pytexsubcount}{1}
\stepcounter{section}
{\newpage\clearpage
\lthtmlfigureA{lstlisting1291}%
\begin{lstlisting}[language=C]
void fname_incref(void);
void fname_decref(void);
\end{lstlisting}%
\lthtmlfigureZ
\lthtmlcheckvsize\clearpage}

{\newpage\clearpage
\lthtmlfigureA{lstlisting1294}%
\begin{lstlisting}[language=C]
int fname_n_in(void);
int fname_n_out(void);
\end{lstlisting}%
\lthtmlfigureZ
\lthtmlcheckvsize\clearpage}

{\newpage\clearpage
\lthtmlfigureA{lstlisting1297}%
\begin{lstlisting}[language=C]
const char* fname_name_in(int ind);
const char* fname_name_out(int ind);
\end{lstlisting}%
\lthtmlfigureZ
\lthtmlcheckvsize\clearpage}

{\newpage\clearpage
\lthtmlfigureA{lstlisting1300}%
\begin{lstlisting}[language=C]
const int* fname_sparsity_in(int ind);
const int* fname_sparsity_out(int ind);
\end{lstlisting}%
\lthtmlfigureZ
\lthtmlcheckvsize\clearpage}

{\newpage\clearpage
\lthtmlinlinemathA{tex2html_wrap_inline5483}%
$ \texttt{ncol}+1$%
\lthtmlinlinemathZ
\lthtmlcheckvsize\clearpage}

{\newpage\clearpage
\lthtmlinlinemathA{tex2html_wrap_inline5487}%
$ \texttt{colind}[i]$%
\lthtmlinlinemathZ
\lthtmlcheckvsize\clearpage}

{\newpage\clearpage
\lthtmlinlinemathA{tex2html_wrap_inline5489}%
$ \texttt{colind}[i+1]$%
\lthtmlinlinemathZ
\lthtmlcheckvsize\clearpage}

{\newpage\clearpage
\lthtmlinlinemathA{tex2html_wrap_inline5491}%
$ \texttt{colind}[\texttt{ncol}]$%
\lthtmlinlinemathZ
\lthtmlcheckvsize\clearpage}

{\newpage\clearpage
\lthtmlinlinemathA{tex2html_wrap_inline5493}%
$ \texttt{nnz} \ne \texttt{nrow}*\texttt{ncol}$%
\lthtmlinlinemathZ
\lthtmlcheckvsize\clearpage}

{\newpage\clearpage
\lthtmlfigureA{lstlisting1322}%
\begin{lstlisting}[language=C]
int fname_n_mem(void);
\end{lstlisting}%
\lthtmlfigureZ
\lthtmlcheckvsize\clearpage}

{\newpage\clearpage
\lthtmlfigureA{lstlisting1325}%
\begin{lstlisting}[language=C]
int fname_work(int* sz_arg, int* sz_res, int* sz_iw, int* sz_w);
\end{lstlisting}%
\lthtmlfigureZ
\lthtmlcheckvsize\clearpage}

{\newpage\clearpage
\lthtmlfigureA{lstlisting1328}%
\begin{lstlisting}[language=C]
int fname(const double** arg, double** res,
          int* iw, double* w, int mem);
\end{lstlisting}%
\lthtmlfigureZ
\lthtmlcheckvsize\clearpage}

\stepcounter{chapter}
\stepcounter{section}
{\newpage\clearpage
\lthtmlfigureA{lstlisting1357}%
\begin{lstlisting}[language=Python]
class MyCallback(Callback):
  def __init__(self, name, d, opts={}):
    Callback.__init__(self)
    self.d = d
    self.construct(name, opts)
\par
# Number of inputs and outputs
  def get_n_in(self): return 1
  def get_n_out(self): return 1
\par
# Initialize the object
  def init(self):
     print 'initializing object'
\par
# Evaluate numerically
  def eval(self, arg):
    x = arg[0]
    f = sin(self.d*x)
    return [f]
\end{lstlisting}%
\lthtmlfigureZ
\lthtmlcheckvsize\clearpage}

{\newpage\clearpage
\lthtmlfigureA{lstlisting1362}%
\begin{lstlisting}[language=Python]
# Use the function
f = MyCallback('f', 0.5)
res = f(2)
print res
\end{lstlisting}%
\lthtmlfigureZ
\lthtmlcheckvsize\clearpage}

{\newpage\clearpage
\lthtmlfigureA{lstlisting1365}%
\begin{lstlisting}[language=Matlab]
  classdef MyCallback < casadi.Callback
    properties
      d
    end
    methods
      function self = MyCallback(name, d)
        self@casadi.Callback();
        self.d = d;
        construct(self, name);
      end
\par
function v=get_n_in(self)
        v=1;
      end
      function v=get_n_out(self)
        v=1;
      end
\par
function init(self)
        disp('initializing object')
      end
\par
function arg = eval(self, arg)
        x = arg{1};
        f = sin(self.d * x);
        arg = {f};
      end
    end
  end
\end{lstlisting}%
\lthtmlfigureZ
\lthtmlcheckvsize\clearpage}

{\newpage\clearpage
\lthtmlfigureA{lstlisting1371}%
\begin{lstlisting}[language=Matlab]
f = MyCallback('f', 0.5);
res = f(2);
disp(res)
\end{lstlisting}%
\lthtmlfigureZ
\lthtmlcheckvsize\clearpage}

{\newpage\clearpage
\lthtmlfigureA{lstlisting1374}%
\begin{lstlisting}[language=C++]
#include "casadi/casadi.hpp"
using namespace casadi;
class MyCallback : public Callback {
private:
  // Data members
  double d;
  // Private constructor
  MyCallback(double d) : d(d) {}
public:
  // Creator function, creates an owning reference
  static Function create(const std::string& name, double d,
                         const Dict& opts=Dict()) {
    return Callback::create(name, new MyCallback(d), opts);
  }
\par
// Number of inputs and outputs
  virtual int get_n_in() { return 1;}
  virtual int get_n_out() { return 1;}
\par
// Initialize the object
  virtual void init() {
    std::cout << "initializing object" << std::endl;
  }
\par
// Evaluate numerically
  virtual std::vector<DM> eval(const std::vector<DM>& arg) {
    DM x = arg.at(0);
    DM f = sin(d*x);
    return {f};
  }
};
\end{lstlisting}%
\lthtmlfigureZ
\lthtmlcheckvsize\clearpage}

{\newpage\clearpage
\lthtmlfigureA{lstlisting1386}%
\begin{lstlisting}[language=C++]
int main() {
  Function f = MyCallback::create("f", 0.5);
  std::vector<DM> arg = {2};
  std::vector<DM> res = f(arg);
  std::cout << res << std::endl;
  return 0;
}
\end{lstlisting}%
\lthtmlfigureZ
\lthtmlcheckvsize\clearpage}

\stepcounter{section}
{\newpage\clearpage
\lthtmlfigureA{lstlisting1398}%
\begin{lstlisting}[language=C]
/*CASADIMETA
:fname_N_IN 1
:fname_N_OUT 2
:fname_NAME_IN[0] x
:fname_NAME_OUT[0] r
:fname_NAME_OUT[1] s
:fname_SPARSITY_IN[0] 2 1 0 2
*/
\end{lstlisting}%
\lthtmlfigureZ
\lthtmlcheckvsize\clearpage}

{\newpage\clearpage
\lthtmlfigureA{lstlisting1403}%
\begin{lstlisting}[language=C]
void fname_simple(const double* arg, double* res);
\end{lstlisting}%
\lthtmlfigureZ
\lthtmlcheckvsize\clearpage}

\stepcounter{section}
\stepcounter{chapter}
\stepcounter{section}
{\newpage\clearpage
\lthtmlinlinemathA{tex2html_wrap_inline5529}%
$ p$%
\lthtmlinlinemathZ
\lthtmlcheckvsize\clearpage}

{\newpage\clearpage
\lthtmlinlinemathA{tex2html_wrap_inline5531}%
$ d$%
\lthtmlinlinemathZ
\lthtmlcheckvsize\clearpage}

{\newpage\clearpage
\lthtmlinlinemathA{tex2html_wrap_inline5541}%
$ s$%
\lthtmlinlinemathZ
\lthtmlcheckvsize\clearpage}

{\newpage\clearpage
\lthtmlinlinemathA{tex2html_wrap_inline5543}%
$ \dot{s}$%
\lthtmlinlinemathZ
\lthtmlcheckvsize\clearpage}

{\newpage\clearpage
\lthtmlinlinemathA{tex2html_wrap_inline5545}%
$ q$%
\lthtmlinlinemathZ
\lthtmlcheckvsize\clearpage}

{\newpage\clearpage
\lthtmlinlinemathA{tex2html_wrap_inline5547}%
$ w$%
\lthtmlinlinemathZ
\lthtmlcheckvsize\clearpage}

{\newpage\clearpage
\lthtmlinlinemathA{tex2html_wrap_inline5551}%
$ y$%
\lthtmlinlinemathZ
\lthtmlcheckvsize\clearpage}

{\newpage\clearpage
\lthtmlinlinemathA{tex2html_wrap_inline5576}%
$ \dot{x} =$%
\lthtmlinlinemathZ
\lthtmlcheckvsize\clearpage}

{\newpage\clearpage
\lthtmlinlinemathA{tex2html_wrap_inline5577}%
$ (t,w,x,s,z,u,p,d)$%
\lthtmlinlinemathZ
\lthtmlcheckvsize\clearpage}

{\newpage\clearpage
\lthtmlinlinemathA{tex2html_wrap_inline5579}%
$ (t,w,x,s,z,u,p,d,\dot{s}) =0$%
\lthtmlinlinemathZ
\lthtmlcheckvsize\clearpage}

{\newpage\clearpage
\lthtmlinlinemathA{tex2html_wrap_inline5581}%
$ (t,w,x,s,z,u,p,d) = 0$%
\lthtmlinlinemathZ
\lthtmlcheckvsize\clearpage}

{\newpage\clearpage
\lthtmlinlinemathA{tex2html_wrap_inline5583}%
$ \dot{q} =$%
\lthtmlinlinemathZ
\lthtmlcheckvsize\clearpage}

\stepcounter{section}
{\newpage\clearpage
\setcounter{equation}{0}
\lthtmldisplayA{subequations5596}%
\setcounter{equation}{-1}
\begin{subequations}\begin{align}  \dot{h} &= v,                    \qquad &h(0) = 0 \\\dot{v} &= (u - a \, v^2)/m - g, \qquad &v(0) = 0 \\\dot{m} &= -b \, u^2,            \qquad &m(0) = 1 \end{align}\end{subequations}%
\lthtmldisplayZ
\lthtmlcheckvsize\clearpage}

{\newpage\clearpage
\lthtmlinlinemathA{tex2html_wrap_inline5598}%
$ u$%
\lthtmlinlinemathZ
\lthtmlcheckvsize\clearpage}

{\newpage\clearpage
\lthtmlinlinemathA{tex2html_wrap_inline5600}%
$ (a,b)$%
\lthtmlinlinemathZ
\lthtmlcheckvsize\clearpage}

{\newpage\clearpage
\lthtmlfigureA{lstlisting1447}%
\begin{lstlisting}[language=Python]
# Python
dae = DaeBuilder()
# Add input expressions
a = dae.add_p('a')
b = dae.add_p('b')
u = dae.add_u('u')
h = dae.add_x('h')
v = dae.add_x('v')
m = dae.add_x('m')
# Add output expressions
hdot = v
vdot = (u-a*v**2)/m-g
mdot = -b*u**2
dae.add_ode(hdot)
dae.add_ode(vdot)
dae.add_ode(mdot)
# Specify initial conditions
dae.set_start('h', 0)
dae.set_start('v', 0)
dae.set_start('m', 1)
# Add meta information
dae.set_unit('h','m')
dae.set_unit('v','m/s')
dae.set_unit('m','kg')
\end{lstlisting}%
\lthtmlfigureZ
\lthtmlcheckvsize\clearpage}

{\newpage\clearpage
\lthtmlfigureA{lstlisting1452}%
\begin{lstlisting}[language=Matlab]
dae = DaeBuilder;
a = dae.add_p('a');
b = dae.add_p('b');
u = dae.add_u('u');
h = dae.add_x('h');
v = dae.add_x('v');
m = dae.add_x('m');
hdot = v;
vdot = (u-a*v^2)/m-g;
mdot = -b*u^2;
dae.add_ode(hdot);
dae.add_ode(vdot);
dae.add_ode(mdot);
dae.set_start('h', 0);
dae.set_start('v', 0);
dae.set_start('m', 1);
dae.set_unit('h','m');
dae.set_unit('v','m/s');
dae.set_unit('m','kg');
\end{lstlisting}%
\lthtmlfigureZ
\lthtmlcheckvsize\clearpage}

\stepcounter{section}
{\newpage\clearpage
\lthtmlfigureA{lstlisting1472}%
\begin{lstlisting}[language=Python]
from pymodelica import compile_jmu
jmu_name=compile_jmu('ModelicaClass.ModelicaModel', \
  ['file1.mo','file2.mop'],'auto','ipopt',\
  {'generate_xml_equations':True, 'generate_fmi_me_xml':False})
\end{lstlisting}%
\lthtmlfigureZ
\lthtmlcheckvsize\clearpage}

{\newpage\clearpage
\lthtmlfigureA{lstlisting1477}%
\begin{lstlisting}[language=Python]
from zipfile import ZipFile
sfile = ZipFile(jmu_name','r')
mfile = sfile.extract('modelDescription.xml','.')
\end{lstlisting}%
\lthtmlfigureZ
\lthtmlcheckvsize\clearpage}

{\newpage\clearpage
\lthtmlfigureA{lstlisting1481}%
\begin{lstlisting}[language=Python]
dae = DaeBuilder()
ocp.parse_fmi('modelDescription.xml')
\end{lstlisting}%
\lthtmlfigureZ
\lthtmlcheckvsize\clearpage}

\stepcounter{section}
{\newpage\clearpage
\lthtmlfigureA{lstlisting1489}%
\begin{lstlisting}[language=Python]
# Python
ocp.make_explicit()
\end{lstlisting}%
\lthtmlfigureZ
\lthtmlcheckvsize\clearpage}

{\newpage\clearpage
\lthtmlfigureA{lstlisting1494}%
\begin{lstlisting}[language=Matlab]
ocp.make_explicit();
\end{lstlisting}%
\lthtmlfigureZ
\lthtmlcheckvsize\clearpage}

\stepcounter{section}
{\newpage\clearpage
\lthtmlfigureA{lstlisting1511}%
\begin{lstlisting}[language=Python]
# Python
f = dae.create('f',\
     ['x','u','p'],['ode'])
\end{lstlisting}%
\lthtmlfigureZ
\lthtmlcheckvsize\clearpage}

{\newpage\clearpage
\lthtmlfigureA{lstlisting1516}%
\begin{lstlisting}[language=Matlab]
f = dae.create('f',...
     {'x','u','p'},{'ode'});
\end{lstlisting}%
\lthtmlfigureZ
\lthtmlcheckvsize\clearpage}

{\newpage\clearpage
\lthtmlfigureA{lstlisting1523}%
\begin{lstlisting}[language=Python]
# Python
f = dae.create('f',\
     ['x','u','p'],\
     ['jac_ode_x'])
\end{lstlisting}%
\lthtmlfigureZ
\lthtmlcheckvsize\clearpage}

{\newpage\clearpage
\lthtmlfigureA{lstlisting1528}%
\begin{lstlisting}[language=Matlab]
f = dae.create('f',...
     {'x','u','p'},
     {'jac_ode_x'});
\end{lstlisting}%
\lthtmlfigureZ
\lthtmlcheckvsize\clearpage}

{\newpage\clearpage
\lthtmlfigureA{lstlisting1536}%
\begin{lstlisting}[language=Python]
# Python
dae.add_lc('gamma',['ode'])
hes = dae.create('hes’,\
  ['x','u','p','lam_ode'],\
  ['hes_gamma_x_x'])
\end{lstlisting}%
\lthtmlfigureZ
\lthtmlcheckvsize\clearpage}

{\newpage\clearpage
\lthtmlfigureA{lstlisting1541}%
\begin{lstlisting}[language=Matlab]
dae.add_lc('gamma,{'ode'});
hes = dae.create(’hes’,...
  {'x','u','p','lam_ode'},...
  {'hes_gamma_x_x'});
\end{lstlisting}%
\lthtmlfigureZ
\lthtmlcheckvsize\clearpage}

\stepcounter{chapter}
\stepcounter{section}
{\newpage\clearpage
\lthtmldisplayA{displaymath5625}%
\begin{displaymath}\begin{array}{lc} \begin{array}{l} \text{minimize:} \\x(\cdot) \in \mathbb{R}^2, \, u(\cdot) \in \mathbb{R} \end{array} \quad \displaystyle \int_{t=0}^{T}{(x_0^2 + x_1^2 + u^2) \, dt} \\\\\text{subject to:} \\\\\begin{array}{ll} \left\{ \begin{array}{l} \dot{x}_0 = (1-x_1^2) \, x_0 - x_1 + u \\\dot{x}_1 = x_0 \\-1.0 \le u \le 1.0, \quad x \ge -0.25 \end{array} \right. & \text{for $0 \le t \le T$} \\x_0(0)=0, \quad x_1(0)=1, \end{array} \end{array}\end{displaymath}%
\lthtmldisplayZ
\lthtmlcheckvsize\clearpage}

{\newpage\clearpage
\lthtmlinlinemathA{tex2html_wrap_inline5627}%
$ T=10$%
\lthtmlinlinemathZ
\lthtmlcheckvsize\clearpage}

\stepcounter{section}
\stepcounter{section}
\stepcounter{section}
\stepcounter{chapter}
\stepcounter{section}
\stepcounter{section}

\end{document}
